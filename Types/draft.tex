\documentclass[10pt,a4paper]{article}
\usepackage[utf8]{inputenc}
\usepackage[russian]{babel}
\usepackage[OT1]{fontenc}
\usepackage{amsmath}
\usepackage{amsfonts}
\usepackage{amssymb}
\usepackage{graphicx}
\usepackage{listings}
\usepackage{mathtools}
\usepackage{tablefootnote}
\usepackage{amsthm}

\theoremstyle{definition}
\newtheorem{Def}{Определение}
\newtheoremstyle{Def}% <name>
{3pt}% <Space above>
{3pt}% <Space below>
{\normalsize}% <Body font>
{}% <Indent amount>
{\itshape}% <Theorem head font>
{:}% <Punctuation after theorem head>
{.5em}% <Space after theorem headi>
{}% <Theorem head spec (can be left empty, meaning `normal')>


\lstdefinestyle{mystyle}{    
    breaklines=true,                 
    captionpos=b,                    
    keepspaces=true,                 
    numbers=right,                    
    numbersep=5pt,                  
    showspaces=false,                
    showstringspaces=false,
    showtabs=false,                  
    tabsize=4
}
 
\lstset{style=mystyle}
\date{\today}
\author{Никита Юрченко}
\title{Про Типы}
\begin{document}
\pagenumbering{gobble} 
\maketitle
\newpage
\pagenumbering{arabic}  

\section{Глава}
Mda kek \\
\subsection{Изоморфизм Карри - Говарда - Ламбека}

Он же Computational Trinitarianism Харпера.\\
Под логической частью имеется ввиду интуиционистское исчисление предикатов в BHK интерпретации. Типы приведены а-ля Хаскелл/Агда, отдельно надо будет ввести для просто типизированного лямбда исчисления (STLC), потому что получится громоздко. Категорная семантика Ловировская \cite{adjointnessLawvere}.\\ \\
\begin{tabular}{ | c | c | c | }
  \hline
  Curry & Howard & Lambek \\
  Proof Theory & Type Theory & Category Theory \\ \hline
  Высказывание А & Тип А & Обьект А \\
  Доказательство А & $\Gamma \vdash a : A$ & $ \Gamma \xrightarrow{\text{a}} A$ \\
  $ A \wedge B $ & Pair A B & $A \times B $ произведение \\
  $ A \vee B$ & Either A B & $A + B $ копроизведение \\
  $ A \supset B $ & $ A \rightarrow B $ & $B^A $ экспоненциал \\
  $\neg A (i. e. A \rightarrow \bot)$ & $ A \rightarrow \bot$ & ?  \\
  $\top$ true & $ \top $ unit type  & $1$ конечный объект \\
  $\bot$ false & $ \bot$ void type  & $0$ начальный объект \\
  \hline
  $\forall x \in A . B(x)$ & $\prod_{x:a}^{} B(x)$ & ? \\
  $\exists x \in A . B(x)$ & $\sum_{x:a}^{} B(x)$ & ? \\
  \hline
  индукция & индуктивный тип (напр. $\mathbb{N}$) & начальная алгебра эндофунктора \\
  \hline
  закон Пирса  & Продолжения  & ? \\
  $ ((P \rightarrow Q) \rightarrow P) \rightarrow P$ & & \\
  \hline
  Трансляция Гливенко & Continuation-passing style & Лемма Йонеды \\
  \hline
  ... & ... & ... \\
  \hline
\end{tabular}
\\
\\
\begin{Def}
Трансляция Гливенко (1929) ($alias$ Теорема Гливенко, negative translation, double-negation translation) :  Пропозициональная формула $\phi$ - классическая тавтология если и только если $\neg \neg \phi$ - интуиционистская тавтология. Расширена до логики первого порядка в виде расширений Куроды и Геделя-Генцена\\
\end{Def}

\subsection{Понятие типа}
По-видимому, не имеет смысла говорить о типах вообще, вне конкретной формальной системы. Пока мне удалось найти 2 основных взгляда на типы:
\begin{enumerate}
\item Типы как множества. \\
  Наиболее распространенный взгляд, особенно когда говорят об основаниях математики и сферической ТТ в вакууме.
\item Типы как приписки к термам. \\
  Распространено в литературе по лямбда исчислению и всему что к нему примыкает. Также популярна у имплементоров языков программирования.
\end{enumerate}

\subsection{Виды равенства}

Здесь много разночтений и философии, привожу по \cite{PML1980}.


\begin{enumerate}
\item Интенсиональное (alias definitional, judgmental)\\
  \quotation{
    Definitional equality is intensional equality, or equality of *meaning* (synonymy). Definitional equality == is a relation between *linguistic expressions*. ... it should not be confused with equality between objects.
    \\ \\ Per - Martyn - L{\"o}f
  }

\end{enumerate}

\newpage


\begin{thebibliography}{9}

\bibitem{PML1980} 
Per Martin-Löf : Intuitionistic Type Theory

\bibitem{adjointnessLawvere} 
Lawvere : Adjointness in Foundations
\\\texttt{http://www.tac.mta.ca/tac/reprints/articles/16/tr16.pdf}

\end{thebibliography}
 
\end{document}

\end{document}