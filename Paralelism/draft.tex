\documentclass[12pt,a4paper,draft]{article}
\usepackage[utf8]{inputenc}
\usepackage[russian]{babel}
\usepackage[OT1]{fontenc}
\usepackage{amsmath}
\usepackage{amsfonts}
\usepackage{amssymb}
\usepackage{graphicx}
\author{Nick Yurchenko}
\title{Paralellism and Concurrency}
\begin{document}
\section{First Section}
(Конспект книги паралельный и конкуррентный хаскелл товарища Симона Марлоу)\\

Во многих случаях не делают различия между паралеллизмом и многопоточностью.\\

Parallelism - разделяем работу меджду несколькими раздельными вычислительными единицами (процессоры, ядра, ГПУ) в надежде что расходы на поддержку параллелизма будут ниже профитов от него\\

Concurrency - она же многопоточность - исполнение в одной программе нескольких потоков управления. Потоки выполняются одновременно в том смысле что побочные эффекты потоков перемежаются между собой. Количество вычислительных ядер не имеет значения. Используется в задачах ИО (считывание - запись БД и пр). Происходит ли выполнение действительно одновременно - остается вопросом реализации.\\

В Хаскеле нет и не может быть никакого "Потока управления" (нет побочных эффектов, порядок выполнения не имеет никакого значения)\\

Единственный профит от Паралелизма - скорость. Многопоточность же позволяет реализовать например асинхронный ввод-вывод. \\

Детерминистская модель программирования - при каждом запуске получаем одинаковый\\ результат. 
Недетерминистская модель программирования - при каждом запуске разлисный результат. Многопоточные программы недетерминированнные тк взаимодействуют с внешними агентами. Недетерминированные программы намного сложнее тестировать.\\

Лучше всего писать детерминированные паралельные программы. Большиинство процессоров предлагают детерминированный паралеллизм в виде конвейерного исполнения (pipelining) и множественных единиц исполнения. \\

Старая нерешаемая проблема автоматического паралелизма (см также закон Амдала) заключается в том, что даже в функциональных ЯП компилятор не знает как "разрезать" программу так чтобы затраты на распаралеливание не сводили на нет профит от паралельных вычислений.\\

\end{document}