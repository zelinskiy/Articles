\documentclass[10pt,a4paper]{article}
\usepackage[utf8]{inputenc}
\usepackage[russian]{babel}
\usepackage[OT1]{fontenc}
\usepackage{amsmath}
\usepackage{amsfonts}
\usepackage{amssymb}
\usepackage{graphicx}
\usepackage{listings}
\usepackage{mathtools}
\usepackage{tablefootnote}
\usepackage{amsthm}

\theoremstyle{definition}
\newtheorem{Def}{Определение}
\newtheoremstyle{Def}% <name>
{3pt}% <Space above>
{3pt}% <Space below>
{\normalsize}% <Body font>
{}% <Indent amount>
{\itshape}% <Theorem head font>
{:}% <Punctuation after theorem head>
{.5em}% <Space after theorem headi>
{}% <Theorem head spec (can be left empty, meaning `normal')>


\lstdefinestyle{mystyle}{    
    breaklines=true,                 
    captionpos=b,                    
    keepspaces=true,                 
    numbers=right,                    
    numbersep=5pt,                  
    showspaces=false,                
    showstringspaces=false,
    showtabs=false,                  
    tabsize=4
}
 
\lstset{style=mystyle}
\date{\today}
\title{Hask}
\begin{document}
\pagenumbering{gobble} 
\maketitle
\newpage
\pagenumbering{arabic}

\section{Plan}
\begin{itemize}
\item Теория множеств
  \begin{enumerate}
  \item Наивная vs Аксиоматическая теория множеств
  \item Система Цермело - Френкеля
  \item Системы Геделя-Бернайса, Куайна, Тарского, фон Неймана, Гротендика
  \item Операции над множествами (Обьединение, Пересечение, Дополнение, Разность, Амальгамированная сумма, Декартово произведение, Симметрическая разность, Булеан, Мощность, законы Де Моргана)
  \item Отношения и их свойства (симметричность, рефлексивность транзитивность, дистрибутивность, латинский квадрат, замыкания (алг. Флойда), аксиома выбора, пары, функции, иньективность, сюрьективность, биективность)
  \item Теория порядков (Предпорядок, частичный, линейный, квази, well, отношение эквивалентности, диаграммы Хассе, грани, факторизация)
  \item Бесконечные множества (Континуум гипотеза, действительные числа, вполне упорядоченные множества, ординалы, арифметика ординалов)
    
  \end{enumerate}
\item Формальные языки и грамматики
  \begin{enumerate}
  \item Иерархия Хомского
  \item КС-грамматики
  \item Нормальная форма Хомского, приведение к ней
  \item Алгоритмы  Кока-Янгера-Касами, Эрли    
  \end{enumerate}  
\item Лямбда исчисление
  \begin{enumerate}
  \item Безтиповое лямбда исчисление
  \item $\alpha$ эквивалентность
  \item $\beta$ редукция (Эквивалентность, Теорема Черча - Россера, Стратегии)
  \item Комбинаторы (базисы, эквивалентность лямбде, полнота по тьюрингу)
  \item Нормальные формы (Бета НФ, заголовочная, слабая заголовочная)
  \item Кодирование чисел, логики, рекурсии, арифметики
  \item Нотация де Брауна
  \item Просто типизированное лямбда исчисление
  \item Проверка и вывод типов в STLC
  \item Система F
  \item Алгоритм Хиндли - Милнера
  \item *Виды. System F$\omega$.    
  \item *Зависимые, линейные, аффинные типы
  \end{enumerate}
\item Алгебраические структуры
  \begin{enumerate}
  \item Теория категорий
  \item Полугруппа
  \item Моноид
  \item Функтор
  \item Аппликатив
  \item Монада
  \item Комонада
  \item Группа
  \item Решетка
  \item *Декартово замкнутые категории и топосы
  \end{enumerate}
\item Haskell
  \begin{enumerate}
  \item Зипперы
  \item Доказательство по эквивалентности
  \item Доказательство по (структурной) индукции
  \item Доказательсва по корекурсии   
  \item System FC  
  \item Ускорение
  \item Параллельные вычисления
    
  \item *Веб (Yesod, Servant, Purescript)
  \item *Схемы рекурсии
  \item *Линзы
  \item *Стрелки
  \item *LiquidHaskell и QuickCheck
  \item *Ленивое перепиcывание графов    
  \end{enumerate}
\item Логика
  \begin{enumerate}
  \item Аксиоматический метод. Реформа Тарского
  \item Гильбертовские системы
  \item Генценовская система LK
  \item Безконтектстные системы натурального вывода
  \item Эквивалентность STLC и PPIL
  \item Генценовская система LJ
  \item Алгебры Гейтинга
  \item Модели Крипке
  \item Расширение LK и LJ предикатами
  \item Эквивалентность System F и LK$\forall$
  \item *Формальная арифметика (Пеано, Сколема, Робинсона, Бюхи, Гейтинга)
  \item *Теория типов Рассела
  \item *Теория типов Мартин - Лефа    
  \item *Пруверы
  \end{enumerate}

\item 
  
 
  
\end{itemize}


\end{document}