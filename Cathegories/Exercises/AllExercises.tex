\documentclass[12pt,a4paper]{article}
\usepackage[utf8]{inputenc}
\usepackage[russian]{babel}
\usepackage[OT1]{fontenc}
\usepackage{mathtools}
\usepackage{ucs}
\usepackage{amsmath}
\usepackage{amsfonts}
\usepackage{amssymb}
\usepackage{graphicx}
\author{Nick Yurchenko}
\title{Сборник упражнений по теории категорий}


\newcommand\defeq{\mathrel{\stackrel{\makebox[0pt]{\mbox{\normalfont\tiny def}}}{=}}}

\begin{document}
\maketitle
\vspace{5cm}
\section{Общие Категории}
\begin{enumerate}
\item Выберите любую категорию по своему выбору и докажите для нее аксиомы категории.

\item Убедитесь в существовании категории $С$ с одним обьектом $\ast$ и множеством морфизмов $C(\ast , \ast)$, которое состоит из алгебраических выражений $t::=x_0|f(t)|g(t,t)$ где $x_0$  - заданная переменная, f и g - заданные символы функций, композицией является подстановка $t[t'/x_0]$ (где $"t'$ замещает $x_0"$ в t определяется рекурсивно).

\item Убедитесь, что любой моноид (М,b,e) является категорией с единственным обьектом $\ast$ и $C(\ast , \ast) \defeq M$

\item Если X и Y - предпорядки, убедитесь что и декартово произведение $X \times Y$ упорядочено в координатном порядке (coordinate-wise)

\item Пусть даны категории C и D, обьекты в категории $C \times D$ это пары (A,B) обьектов из C и D соответственно. Проверьте, что не существует совершенно очевидной категории $C \times D$. Сравните с предыдущим вопросом, теперь учитывая что X и Y являются категориями.

\item Если $f : X \rightarrow Y$ и $g : Y \rightarrow Z$ являются монотонными функциями для предпорядков, то их композиция тоже $f \circ g : X \rightarrow Z$, которая определена как $(g \circ f)(x) \defeq g(f(x))$ для всех $x \in  X$. Удостоверьтесь в этом, и в вытекающем отсюда факте, что предпорядки формируют категорию

\item $\star$ Пускай обьектами в категории С являются пары (A,R) где А- множество, R - бинарное отношение, заданное на А: $R \subseteq A \times A$. Найдите морфизмы в этой категории.

\section{Изоморфизм, Эпиморфизм, Мономорфизм, Инициальный и Терминальный обьекты}

\item Докажите, что в категории множеств эпиморфизмы являются сюрьективными функциями

\item Докажите, что в категории множеств мономорфизмы являются иньективными функциями

\item Показать, что если f и g являются мономорфизмами, то их композиция $g \circ f$ тоже является мономорфизмом

\item Показать, что если композиция $g \circ f$ является мономорфизмом, то f является мономорфизмом

\item Показать, что если f и g являются эпиморфизмами, то их композиция $g \circ f$ тоже является эпиморфизмом

\item Показать, что если композиция $g \circ f$ является эпиморфизмом, то g является эпиморфизмом

\item Найти категорию, в которой содержится морфизм, являющийся одновременно мономорфизмом и эпиморфизмом, но не изоморфизмом.

\item ПОказать что если $f^{-1}$ является обратным $f : A \rightarrow B$ и $g^{-1}$ является обратным $g : B \rightarrow C$ , то $f^{-1} \circ g^{-1}$ является обратным по отношению к $g \circ f$
\item Показать, что терминальные обьекты попарно изоморфны

\item Показать, что инициальные обьекты попарно изоморфны

\item Найти категорию без инициального обьекта
\item Найти категорию без терминального обьекта
\item Найти категорию, в которой один обьект является и инициальным и терминальным одновременно

\section{Разновидности пределов}
\item Показать, что $\langle f \circ h , g \circ h \rangle = \langle f,g \rangle \circ h$
\item Показать, что $(f \times h) \circ \langle g,k \rangle = \langle f \circ g , h \circ k \rangle $
\item Показать, что $(f \times h) \circ (g \times k) = (f \circ g) \times (h \circ k)$

\item Пускай X и Y - обьекты в категории частично упорядоченных множеств (poset). Чем будет являться их произведение?

\item Пускай X и Y - обьекты в категории частично упорядоченных множеств (poset). Чем будет являться их копроизведение?



\end{enumerate}




\end{document}
